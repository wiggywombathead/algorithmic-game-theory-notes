\section{Network Routing games}
We turn our attention now to network flows, and begin by defining the following
model. Let $G=(V,E)$ be a directed graph with:
\begin{itemize}
	\item $k$ source-destination pairs $(s_i, t_i)$, each with demand $r_i \in \mathbb{R}_{\ge 0}$.
	\item latency functions $\ell_e : \mathbb{R}_{\ge 0} \rightarrow
		\mathbb{R}_{\ge 0}$ for all edges, mapping traffic to latency
	\item $P_i$, the set of paths from $s_i$ to $t_i$ for each
		source-destination pair
	\item Let $P := \bigcup^k P_i$
	\item Flow $f : P \rightarrow \mathbb{R}_{\ge 0}$ such that $\sum_{p \in
		P_i} f(p) = r_i$
	\item Flow on edge $e$, $f_e := \sum_{p \in P_i : e \in p} f(p)$
	\item Latency of path $p$ $L_p (f) := \sum_{e \in p} \ell_e (f_e)$
	\item Social cost of flow $C(f) = \sum_{p \in P} f(p) \cdot L_p(f) =
		\sum_{e \in E} f(e) \cdot \ell_e (f_e)$
\end{itemize}

Consider the following example:

TODO: diagram

We have:
\begin{itemize}
	\item $\ell_A(x) = x$, $\ell_B(x) = 1$, $\ell_C(x) = 1$, $\ell_D(x) = x$
	\item $P = P_1 = \{ (A,B), (C,D) \}$
\end{itemize}

Call $(A,B)$ the ``up'' path, and $(C,D)$ the ``down'' path. Consider the flow
$f$ for $\lambda \in [0,1]$ where $f^\lambda(\text{up}) = \lambda$ and
$f^\lambda(\text{down}) = 1 - \lambda$. We have the following flows on edges:
\begin{equation*}
	\begin{gathered}
		f_A^\lambda = f_B^\lambda = \lambda \\
		f_C^\lambda = f_D^\lambda = 1 - \lambda
	\end{gathered}
\end{equation*}

The latencies of the paths are:
\begin{equation*}
	\begin{gathered}
		L_\text{up}(f^\lambda) = \ell_A(f_A^\lambda) + \ell_B(f_B^\lambda) = \ell_A(\lambda) + \ell_B(\lambda) = \lambda + 1 \\
		L_\text{down}(f^\lambda) = \ell_C(f_C^\lambda) + \ell_D(f_D^\lambda) = \ell_C(1 - \lambda) + \ell_D(1 - \lambda) = 1 + 1 - \lambda = 2 - \lambda
	\end{gathered}
\end{equation*}

The social cost of flow $f^\lambda$ is hence given by:
\begin{equation*}
	\begin{split}
		C(f^\lambda) & = \sum_{p\in P} f^\lambda (p) \cdot L_p(f^\lambda) \\
		& = f^\lambda (\text{up}) \cdot L_\text{up} (f^\lambda) + f^\lambda (\text{down}) \cdot L_\text{down}(f^\lambda) \\
		& = \lambda (1 + \lambda) + (1 - \lambda) (2 - \lambda) \\
		& = 2(\lambda^2 - \lambda + 1) \\
	\end{split}
\end{equation*}

The cost of this flow is minimised by setting $\lambda = 1/2$, which gives a
social cost of $C(f^\frac{1}{2}) = 2(\frac{1}{2}^2 - \frac{1}{2} + 1) = 3/2$.

\subsection{Non-atomic flow games}

So far there has been no game to play -- we could simply compute the
cost-minimising flow and route the traffic as such. However, in selfish routing
we assume that the agents are responsible for routing the traffic, each trying
to minimise his own latency. In this model, there are:
\begin{itemize}
	\item infinitely many player $i \in [0, r]$, each controlling an
		infinitesimal amount of traffic
	\item each player aims to minimise their own latency
\end{itemize}

\textbf{Wardrop's First Principle:} if a route is used, then it is at least as
good as all other routes (otherwise they would simply take another better
route).

\begin{definition}[Wardrop Equilibrium]
	Flow $f$ is a Wardrop Equilibrium (WE) if for every $i \in [k]$, for
	every pair of paths $p, p' \in P_i$, if $f(p) > 0$ then $L_p(f) \le
	L_p'(f)$. In other words, whenever there is some flow on some path $p$,
	then all other alternative paths $p'$ have at least the same latency.
\end{definition}

\begin{fact}
	Every flow network has a Wardrop Equilibrium.
\end{fact}

\subsubsection{Example}

\begin{claim}
	The only Wardrop Equilibrium in the previous game is the flow
	$f^\frac{1}{2}$
\end{claim}
\begin{proof}
	First observe that setting $\lambda = 1/2$ is indeed an equilibrium: the
	flow on the ``up'' path is $\lambda + 1$ = 3/2, equal to the flow on the
	``down'' path, $2 - \lambda = 3/2$. Therefore no player has the incentive
	to deviate to the other path, as it would not decrease the latency they
	experience.

	If $\lambda > 1/2$ then the flow on the upper path is non-zero. But
	$L_\text{up}(f^\lambda) = 1 + \lambda > 3/2$, while
	$L_\text{down}(f^\lambda) = 2 - \lambda < 3/2$, so $L_\text{up}(f^\lambda)
	> L_\text{down}(f^\lambda)$, meaning players on the top path would want to
	move to the bottom path.

	If $\lambda < 1/2$ then the flow on the bottom path is non-zero. Then we
	have $L_\text{down}(f^\lambda) > 3/2$ and $L_\text{up}(f^\lambda) < 3/2$,
	so players on the bottom path would want to switch to the top path. Hence
	the flow $f^\frac{1}{2}$ is the only equilibrium.
\end{proof}

\subsubsection{Braess' Paradox}

TODO: diagram

We have installed a bridge $E$ that incurs no latency when travelling over it,
that is, $\ell_E(x) = 0$.

We have the paths $P = \{ (A,B), (A,E,D), (C,D) \}$ which we will label ``up'',
``bridge'', and ``down'' respectively. We consider the flow $f^{\conj{\lambda}}
= f^{\lambda_1, \lambda_2, \lambda_3}$ that sends $\lambda_1$ units of flow
through the top path, $\lambda_3$ units through the bottom path, and the
remaining $\lambda_2 = 1 - \lambda_1 - \lambda_3$ units through the bridge
path. The flows are:
\begin{equation*}
	\begin{split}
		f^{\conj{\lambda}}_A & = \lambda_1 + \lambda_2 \\
		f^{\conj{\lambda}}_B & = \lambda_1 \\
		f^{\conj{\lambda}}_C & = \lambda_3 \\
		f^{\conj{\lambda}}_D & = \lambda_2 + \lambda_3 \\
		f^{\conj{\lambda}}_E & = \lambda_2 \\
	\end{split}
\end{equation*}

The latencies are:
\begin{equation*}
	\begin{split}
		L_\text{up}(f^{\conj{\lambda}}) & = \lambda_1 + \lambda_2 + 1 =
		2 - \lambda_3 \\
		L_\text{down}(f^{\conj{\lambda}}) & = 1 + \lambda_2 + \lambda_3 = 2 -
		\lambda_1 \\
		L_\text{bridge}(f^{\conj{\lambda}}) & = \lambda_1 + 2\lambda_2 +
		\lambda_3 = 2 - \lambda_1 - \lambda_3
	\end{split}
\end{equation*}

The minimum social cost is again $3/2$, achieved by flow $f^{\frac{1}{2}, 0,
\frac{1}{2}}$. However, the only Wardrop Equilibrium in this game is the flow
$f^{0,1,0}$, where everyone uses the zero-cost bridge.

\begin{claim}
	The only Wardrop Equilibrium in the game is the flow $f^{0,1,0}$.
\end{claim}
\begin{proof}
	The flow $f^{0,1,0}$ is a Wardrop Equilibrium as $L_\text{up}(f^{0,1,0}) =
	L_\text{down}(f^{0,1,0}) = L_\text{bridge}(f^{0,1,0}) = 2$.

	Now consider the cases where $\lambda_1, \lambda_3 \neq 0$. In either case
	some flow is sent along the ``bridge'' path. If only $\lambda_1 > 0$:
	\begin{itemize}
		\item there is some flow sent along the ``up'' path, and none along the
			``down'' path
		\item $L_\text{up}(f^{\conj{\lambda}}) = 2 > 2 - \lambda_1 =
			L_\text{bridge}(f^{\conj{\lambda}})$
		\item so players have incentive to deviate to taking the ``bridge''
			path
	\end{itemize}

	If only $\lambda_3 > 0$:
	\begin{itemize}
		\item there is some flow sent along the ``down'' path, and none along the
			``up'' path
		\item $L_\text{down}(f^{\conj{\lambda}}) = 2 > 2 - \lambda_3 =
			L_\text{bridge}(f^{\conj{\lambda}})$
		\item so players have incentive to deviate to taking the ``bridge''
			path
	\end{itemize}

	If both $\lambda_1, \lambda_3 > 0$, then all paths are used, and
	$L_\text{bridge}(f^{\conj{\lambda}}) < L_\text{up}(f^{\conj{\lambda}}),
	L_\text{down}(f^{\conj{\lambda}})$.

	Hence whenever $\lambda_1, \lambda_3 > 0$ players always have an incentive
	to deviate to the ``bridge'' path. Therefore the only Wardrop Equilibrium
	is the flow $f^{0,1,0}$.

\end{proof}

\begin{definition}[Price of Anarchy]
	The Price of Anarchy (POA) of a network $G$ is the maximum ratio of
	the cost of the worst possible Wardrop flow and the cost of the optimal
	feasible flow. That is,
	\begin{equation}
		POA(G) = \frac{ \max_{f \text{ is WE}} C(f) }{ \min_{f^* \text{ is
		flow}} C(f^*) }
	\end{equation}
\end{definition}

\begin{fact}
	The Price of Anarchy in Braess' Network is $\frac{2}{3/2} = 4/3$.
\end{fact}

So the Wardrop Equilibrium flow is $1/3$ worse than the optimal in Braess'
Network. This example is interesting as it shows that adding links that seem to
be beneficial may actually degrade performance if the traffic is controlled by
selfish players. We will now cover an even simpler example where the Price of
Anarchy can be made arbitrarily bad.

\subsubsection{Pigou's Network}

The latency functions are $\ell_A(x)=x$, $\ell_B(x)=1$. Consider the flow
\begin{equation*}
	\begin{split}
		f^\varepsilon : & \, A \mapsto 1 - \varepsilon \\
		& B \mapsto \varepsilon
	\end{split}
\end{equation*}

No matter the value of $\varepsilon$, everyone always wants to take the top
path: sending $\varepsilon$ along edge $B$ incurs a latency of 1, but a latency
of $1-\varepsilon$ along edge $A$. Hence the only Wardrop Equilibrium is $f^0$,
whose social cost is 1. The minimum social cost is achieved by the flow
$f^\frac{1}{2}$, which incurs a social cost of $3/4$. Hence the Price of
Anarchy of the above network is $\frac{1}{3/4} = 4/3$.

Now consider the same network with latency functions $\ell_A(x)=x^d$ for $d \in
[0,1]$, and $\ell_B(x)=1$. The flow $f^0$ is still the only Wardrop Equilibrium
with a social cost of $C(f^0) = 1$.

The cost of sending $\varepsilon$ units of flow along $B$ and the remainder
along $A$ is
$$C(f^\varepsilon) = (1-\varepsilon)(1-\varepsilon)^d + 1 \cdot \varepsilon =
\varepsilon + (1-\varepsilon)^{d+1}$$

This is minimised when the derivative is 0:
\begin{equation*}
	\begin{split}
		1 - (d+1)(1-\varepsilon^*)^d & = 0 \\
		\varepsilon^* = 1 - \frac{1}{\sqrt[d]{d+1}}
	\end{split}
\end{equation*}

The cost of this flow is:
\begin{equation*}
	\begin{split}
		C(f^{\varepsilon^*}) & = \varepsilon^* + (1-\varepsilon^*)^{d+1} \\
		& = 1 - \frac{d}{(d+1)\sqrt[d]{d+1}}
	\end{split}
\end{equation*}

As $d \rightarrow \infty$, $C(f^{\varepsilon^*}) \rightarrow 0$. Therefore the
Price of Anarchy ($\frac{\text{stable}}{\text{optimal}}$) is unbounded.

\subsubsection{Pigou Bound}

The latency functions used in Pigou's example are very steep. Can we derive
better bounds on the Price of Anarchy by restricting the kind of
latency functions that are allowed in the network?

\begin{definition}[Pigou Bound]
	Let $\mathcal{C}$ be a set of latency functions. The Pigou Bound
	$\pi(\mathcal{C})$ is defined as follows:
	\begin{equation}
		\pi(\mathcal{C}) = \sup_{\ell \in \mathcal{C}} \sup_{\varepsilon \in
		[0, r]}
		\frac{r \cdot \ell(r)}{\varepsilon \cdot \ell(\varepsilon) +
		(r-\varepsilon) \cdot \ell(r)}
	\end{equation}
\end{definition}

This value is the worst Price of Anarchy possible in a network with two
parallel edges, one with a constant latency function and the other with a
latency function drawn from a set $\mathcal{C}$. The idea is to take the worst
possible latency function $\ell \in \mathcal{C}$ and the worst possible demand
$r$. One of the two edges has constant latency equal to $\ell(r)$, and the
other's latency is given by the latency function $\ell$.

TODO: diagram

\begin{theorem}
	If a class $\mathcal{C}$ of latency functions contains all constant
	functions, then $POA(\mathcal{C}) \le \pi(\mathcal{C})$.
\end{theorem}

Recall that the Price of Anarchy can be made arbitrarily large, i.e.
$POA(\mathcal{C}) \ge \pi(\mathcal{C})$

\begin{corollary}
	The worst POA in Pigou's Network is equal to the worst POA in all graphs.
\end{corollary}

\begin{theorem}[Variational Inequality Characterisation]
	The flow $f$ is a Wardrop Equilibrium iff for every feasible flow $f^*$ we
	have:
	\begin{equation}
		\sum_{e \in E} f_e \cdot \ell_e(f_e) \le \sum_{e \in E} f^*_e \cdot
		l_e(f_e)
	\end{equation}

	That is, the social cost of $f$ is at most the social cost of $f^*$ in the
	network where each edge has constant latency function $\ell_e(f_e)$.
\end{theorem}
\begin{proof}
	Begin by writing the right hand side of the inequality in terms of paths
	rather than edges and define $H_f(f^*)$ as:
	\begin{equation*}
		\begin{split}
			H_f(f^*) & = \sum_{e \in E} f^*_e \cdot \ell_e(f_e) \\
			& = \sum_{i \in [k]} \sum_{p \in P_i} f^*(p) \cdot L_p(f)
		\end{split}
	\end{equation*}

	Note that the right hand side is equal to $H_f(f^*)$ while the left hand
	side is equal to $H_f(f)$.
\end{proof}
