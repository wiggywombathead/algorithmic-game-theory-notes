\section{Two-player zero-sum games}

\label{sec:TPZSgames}

Consider a zero-sum game determined by the matrix $\vect{A} \in \mathbb{R}^{m_1
\times m_2}_{\ge 0}$ which gives the payoffs to player I.  The fact that the
game is zero-sum $\Rightarrow$ payoffs for player II $=\vect{A}$.

\begin{definition}[Maximin]
	A strategy $\vect{x}^* \in \Delta^{m_1}$ for player I is a \textit{maximin}
	strategy if

	\begin{equation}
		\vect{x}^* \in \argmax_{\vect{x} \in \Delta^{m_1}} \left[
			\min_{j \in [m_2]} (\vect{x}^\top \vect{A})_j \right]
	\end{equation}
\end{definition}

\begin{definition}[Minimax]
	A strategy $\vect{y}^* \in \Delta^{m_2}$ for player II is a
	\textit{minimax} strategy if

	\begin{equation}
		\vect{y}^* \in \argmin_{\vect{y} \in \Delta^{m_2}} \left[
			\max_{i \in [m_1]} (\vect{A} \vect{y})_i \right]
	\end{equation}
\end{definition}

A maximin strategy ``maximizes the minimum expected payoff to player I''. A
minimax strategy ``minimises the maximum expected payoff to player II''.
Maximin and minimax values can be thought of as \textit{safety utilities} -- no
matter what the other plays, they can set a minimum payoff they will receive.

\begin{itemize}
	\item \textsc{maximin} -- utility that player I can get if he commits to
		move first
	\item \textsc{minimax} -- utility that plauer II can limit player I's
		utility to if player II commits to move first
\end{itemize}

\begin{fact}
	maximin value $\le$ minimax value
\end{fact}

\begin{theorem}[von Neumann, 1928]
	maximin value = minimax value
\end{theorem}

\subsection{Two-player games as linear programs}

The maximization problem is as follows:

\begin{equation}
	\begin{split}
		\text{maximise}_{\vect{x},v} v \text{ subject to } \vect{x} & \in \Delta^{m_1} \\
		\vect{x}^\top \vect{A} & \ge \begin{pmatrix}
			v \\
			\vdots \\
			v
		\end{pmatrix}
	\end{split}
\end{equation}

Note that this is not yet a Linear Program due to the first constraint. We can
therefore rewrite the optimisation problem as an LP in standard form as:

\begin{equation}
	\begin{split}
		\text{maximise}_{\vect{x},v} v \text{ subject to }
		-\vect{x}^\top \vect{A} + \begin{pmatrix}
			v \\
			\vdots \\
			v
		\end{pmatrix} & \le 0 \\
		\mathds{1}^\top \vect{x} & \le 1 \\
		\vect{x} & \ge 0
	\end{split}
\end{equation}

The primal LP formulation is the following:

\begin{equation}
	\begin{split}
		\text{maximise}_{\vect{x},v}
		(0,\ldots,0,1) \begin{pmatrix}
			x_1 \\
			\vdots \\
			x_{m_1} \\
			v
		\end{pmatrix}
		\text{ subject to }
		\begin{pmatrix}
			& & & 1 \\
			& & & \vdots  \\
			\multicolumn{3}{c}
			{\raisebox{\dimexpr\normalbaselineskip+.7\ht\strutbox-.5\height}[0pt][0pt]
			{{$-\vect{A}^\top$}}} & 1 \\
			1 & \cdots & 1 & 0
		\end{pmatrix} \begin{pmatrix}
			x_1 \\
			\vdots \\
			x_{m_1} \\
			v
		\end{pmatrix} & \le \begin{pmatrix}
			0 \\
			\vdots \\
			0 \\
			1
		\end{pmatrix} \\
		\begin{pmatrix}
			x_1 \\
			\vdots \\
			x_{m_1} \\
			v
		\end{pmatrix} & \ge 0
	\end{split} 
\end{equation}

The dual LP is:

\begin{equation}
	\begin{split}
		\text{minimise}_{\vect{y},w}
		(0,\ldots,0,1) \begin{pmatrix}
			y_1 \\
			\vdots \\
			y_{m_2} \\
			w
		\end{pmatrix}
		\text{ subject to }
		\begin{pmatrix}
			& & & 1 \\
			& & & \vdots  \\
			\multicolumn{3}{c}
			{\raisebox{\dimexpr\normalbaselineskip+.7\ht\strutbox-.5\height}[0pt][0pt]
			{{$-\vect{A}$}}} & 1 \\
			1 & \cdots & 1 & 0
		\end{pmatrix} \begin{pmatrix}
			y_1 \\
			\vdots \\
			y_{m_2} \\
			w
		\end{pmatrix} & \ge \begin{pmatrix}
			0 \\
			\vdots \\
			0 \\
			1
		\end{pmatrix} \\
		\begin{pmatrix}
			y_1 \\
			\vdots \\
			y_{m_2} \\
			w
		\end{pmatrix} & \ge 0
	\end{split} 
\end{equation}

\subsection{LP duality and zero-sum games}

We have the primal (P):

\begin{equation}
	\begin{split}
		\text{max}_{\vect{x}, v} v \text{ subject to }
		\vect{A}^\top \vect{x} & \ge \vect{v} \\
		\mathds{1}^\top \vect{x} & \le 1 \\
		\vect{x} & \ge 0 \\
		\vect{v} & \ge 0
	\end{split}
\end{equation}

We denote by $\vect{v}$ (and analogously for $\vect{w}$) the column vector
$\begin{pmatrix} v \\ \vdots \\ v \end{pmatrix}$. The dual (D) is:

\begin{equation}
	\begin{split}
		\text{min}_{\vect{y}, w} v \text{ subject to }
		\vect{A} \vect{y} & \le \vect{w} \\
		\mathds{1}^\top \vect{y} & \le 1 \\
		\vect{y} & \ge 0 \\
		\vect{w} & \ge 0
	\end{split}
\end{equation}

\begin{fact}
	\label{fact:A}
	For every (P) feasible solution $(\vect{x}, v)$ and for every
	mixed strategy $\vect{y} \in \Delta^{m_2}$, $(\vect{x}^\top
	\vect{A})\vect{y} \ge \vect{v}$.
\end{fact}

This says that the expected utility for player I by playing $\vect{x}$, if
player II chooses any $\vect{y}$, is at least $\vect{v}$.

\begin{proof}
	By the definition of the dot product:
	\begin{equation*}
		\vect{x}^\top \vect{A} \vect{y} = \sum_{j \in [m_2]}
		(\vect{x}^\top \vect{A})_j \cdot \vect{y}_j
	\end{equation*}

	By (P)-feasibility, $\vect{A}^\top \vect{x} \ge 0$ and
	$\vect{y}_j \ge 0$, so:
	\begin{equation*}
		\sum_{j \in [m_2]} (\vect{x}^\top \vect{A})_j \cdot
		\vect{y}_j \ge \sum_{j \in [m_2]} v \cdot \vect{y}_j = v
		\sum_{j \in [m_2]} \vect{y}_j = v
	\end{equation*}
\end{proof}

\begin{fact}
	\label{fact:B}
	If $\vect{A} \ge 0$, then for every (P)-optimal solution $(\vect{x}, v)$ we
	have that $\vect{x} \in \Delta^{m_1}$ is a mixed strategy for player I.
\end{fact}

\begin{proof}
	If $\mathds{1} \vect{x} > 1 - \varepsilon$ for some $\varepsilon > 0$, then
	$(\frac{\vect{x}}{1-\varepsilon}, \frac{v}{1-\varepsilon})$ is also
	(P)-feasible. TODO
\end{proof}

\begin{fact}
	\label{fact:C}
	For every (D)-feasible solution $(\vect{y}, w)$ and for every mixed
	strategy $\vect{x} \in \Delta^{m_1}$, $\vect{x}^\top(\vect{A} \vect{y}) \le
	w$.
\end{fact}

\begin{fact}
	\label{fact:D}
	If $\vect{A} \ge 0$, then for every (D)-optimal solution $(\vect{y}, w)$ we
	have that $\vect{y} \in \Delta^{m_2}$ is a mixed strategy for player II.
\end{fact}

\begin{fact}
	(P) has an optimal solution of finite value.
\end{fact}

\begin{proof}
	The feasible set is non-empty: TODO
\end{proof}

\begin{theorem}[Existence and structure of MNE in zero-sum games]
	For every two-player zero-sum game $\vect{A} \in \mathbb{R}^{m_1 \times
	m_2}_{\ge 0}$, there are mixed strategies $\vect{x} \in \Delta^{m_1},
	\vect{y} \in \Delta^{m_2}$ such that:

	\begin{itemize}
		\item $\vect{x}$ and $\vect{y}$ are maximin and minimax strategies,
			respectively
		\item $(\vect{x}, \vect{y})$ is a Nash Equilibrium
		\item all Nash Equilibria have the same expected payoffs (the
			\textnormal{value} of the game)
	\end{itemize}
\end{theorem}

\begin{proof}
	Let $(x, v)$ and $(y, w)$ be optimal solutions of (P) and (D),
	respectively.

	By the Strong Duality Theorem, we have that $v = w$. Using
	Facts~\ref{fact:A} to \ref{fact:D}, we know that $x$ is a maximin strategy
	for player I, and $y$ is a minimax strategy for player II.

	$x$ and $y$ are mutual best responses, so $(x,y)$ is a Nash Equilibrium.
\end{proof}

\begin{corollary}
	Equilibrium computation in two-player zero-sum games is computable in
	polynomial time.
\end{corollary}
