\section{PPAD}
\begin{definition}[Search problem]
	An algorithmic problem is a total search problem if each instance $x$ has a
	search space $S_x \subseteq \{0,1\}*$ of bitstrings of length polynomial in
	$|x|$, as well as a non-empty subset $Q_x$ of valid solutions.
\end{definition}

Given the input $x$ to some search problem, we wish to find a $y \in Q_x$. For
example, the computational problem \textsc{Nash} is, given a finite game in
strategic form and parameter $\varepsilon > 0$, find an
$\varepsilon$-Nash Equilibrium. By Nash's Theorem, every finite game has a Nash
Equilibrium, hence every finite game also has an $\varepsilon$-Nash
Equilibrium.

The problem \textsc{Sperner} is, given a valid colouring of a
subdivision of a simplex, compute a fully-coloured vertex in such a colouring.

The problem \textsc{$\alpha$-Brouwer} is, given a continuous function $f:S
\rightarrow S$ from a compact convex set to itself and parameter $\alpha > 0$,
find an $\alpha$-approximate fixed point of $f$. 

\subsection{\textsc{End-of-a-Line}}
The problem \textsc{(Explicit) End-of-a-Line} is, given a directed graph
$G=(V,\vec{E})$ such that every vertex has at most one predecessor and at most
one successor, and an initial vertex $v_0$ (the \emph{standard source}) with
in-degree 0 and out-degree 1, find a vertex of out-degree 1 that is not equal
to $v_0$. Recall that the graph we constructed when computing the number of
panchromatic triangles is of such a structure. To solve \textsc{Explicit
End-of-a-Line}, we can simply follow the edges of the graph until we reach a
sink.

Another formulation of the problem is as follows: we are given a graph
$G=(V,\vec{E})$ in which vertices are $k$-bit strings (so $|V|\le 2^k$). The
graph is encoded by two functions $P,S : \{0,1\}^k \rightarrow \{0,1\}^k$ that
compute a node's predecessor and successor, respectively. That is, there is an
edge $(u,v)$ if and only if $S(u)=v$, $P(v)=u$, and $v \neq (0,\ldots,0)$. The
\textsc{End-of-a-Line} problem is to find a node other than $(0,\ldots,0)$ that
either has a successor but no predecessor or has a predecessor but no
successor.

Due to a simple parity argument, we know such a node exists -- the problem is
in finding one. We could just test all nodes in the graph, but since there are
$2^k$ of them this cannot be done in time polynomial in the size of the input.
It is not known whether \textsc{End-of-a-Line} can always be solved in
polynomial time.

\begin{definition}[Polynomial-time reduction]
	A polynomial-time reduction from a search problem $X$ to search problem $Y$
	is a pair $(f,g)$ of polynomial-time computable functions such that $f$
	maps inputs of $X$ to inputs of $Y$, and $g$ maps valid solutions of $Y$ to
	valid solutions of $X$. We write $X \le_m Y$.
\end{definition}

For $x \in X$ we have $f(x) \in Y$, and for any solution $y$ of $Y$ for input
$f(x)$ we have $g(x,y)$ as a solution to $X$ for input $x$.

Informally, the class PPAD (Polynomial Parity Argument (Directed case)) is the
set of all problems whose solution space can be formulated as the set of all
sinks and all nonstandard sources in a directed graph with the properties as
above. Equivalently, PPAD is the class of all search problems $\pi$ such that
$\pi \le_p$ \textsc{End-of-a-Line}. While PPAD is a good metric for
intractability, it is a subset of NP and hence ``not as hard''.

\begin{theorem}
	\textsc{Nash} $\le_p$ \textsc{Brouwer} $\le_p$ \textsc{Sperner} $\le_p$
	\textsc{End-of-a-Line}
\end{theorem}

\begin{theorem}[Goldberg, 2006]
	\textsc{Nash} is PPAD-complete.
\end{theorem}

