\section{Auctions}

\subsection{Single-Item Auctions}

In a single-item auction there are:

\begin{itemize}
	\itemsep0em 
	\item 1 item
	\item $n$ bidders interested in acquiring the item
	\item a private valuation $v_i \in \mathbb{R}$ for obtaining the item
\end{itemize}

The goal is to give the item to the bidder with the highest valuation for it.
As a first attempt at designing such a mechanism, we simply ask each bidder for
their bid $b_i \in \mathbb{R}$ and give the item to the bidder who bids the
highest. Note that since $v_i$ is private, bidders have the option to act
untruthfully. This is a poor mechanism as it can be manipulated -- bidders may
bid arbitrarily high and are not incentivised to report their true valuations.

For our second attempt, we consider the First-Price Auction: first, ask the
bidders to submit a bid $b_i$, give the item to the bidder with the highest bid
and make them pay this bid. Bidders have \emph{quasi-linear utility}, so if the
item is awarded to bidder $i$ they receive utility $u_i = v_i - b_i$, while
everyone else gets 0. This mechanism stops players bidding arbitrarily high, as
their is the risk they actually have to pay that bid.  However, the mechanism
can still be manipulated -- the winner has the incentive to bid $v_j +
\varepsilon$, where $v_j$ is the second highest valuation. This would net them
utility $v_i - (v_j + \varepsilon) = v_i - v_j - \varepsilon > 0$.

\subsubsection{Vickrey's Mechanism}

Vickrey's mechanism works as follows:

\begin{itemize}
	\itemsep0em
	\item Ask for each player's bid $b_i$
	\item Make the player $i$ with the highest bid pay the second
		highest bid $B$ -- the winner gets utility $u_i = v_i - B$,
		while everyone else gets utility 0.
\end{itemize}

\begin{theorem}
	For all bids $b_1, \ldots, b_n$, let $u_i$ be player $i$'s utility if $b_i
	= v_i$, and $u_i'$ otherwise. Then $u_i \ge u_i'$ i.e.  telling the truth
	is a dominant strategy.
\end{theorem}

\begin{proof}
	Assume $i$ wins by bidding $v_i$ (i.e. by being truthful) and let $B$
	denote the second highest bid. Then $u_i = v_i - B \ge 0$, as either $i$
	wins and pays $v_i - B$ where $v_i > B$, or $i$ loses and gets utility 0.
	Suppose $b_i > B$. Player $i$ still wins and pays the same: $u_i' = v_i -
	B$, so there is no incentive to overbid. Now suppose that $b_i < B$. $i$
	would lose, so $u_i' = 0 \le u_i$.

	Now assume $i$ loses by bidding $b_i = v_i$, so $u_i = 0$. Let player $j$
	be the winner, so $b_j \ge v_i$. For $b_i < b_j$, $i$ still loses, so $u_i'
	= 0$. For $b_i > b_j$, $i$ wins and pays $b_j$, so $u_i' = v_i - b_j \le
	0$.

	In all cases, telling the truth by setting $b_i = v_i$ is no worse than
	attempting to manipulate the mechanism, so truth-telling is a dominant
	strategy.
\end{proof}

\begin{definition}[Incentive Compatibility]
	An auction mechanism is incentive-compatible if bidding their true
	valuation is a weakly dominant strategy for each bidder.
\end{definition}

An auction being incentive compatible is a stronger condition than it just
having a Nash Equilibrium. Vickrey's mechanism is incentive compatible for
single-item auctions.

\subsection{Combinatorial Auctions}

In a combinatorial auction there is:

\begin{itemize}
	\itemsep0em
	\item a set $S$ of items to be sold
	\item $n$ bidders
	\item for every bidder $i$ and every subset $S' \subseteq S$,
		$v_i(S') \in \mathbb{R}$ is the valuation of bidder $i$ for set
		$S'$
\end{itemize}

An auction mechanism then:

\begin{itemize}
	\itemsep0em
	\item receives bids $b_1, \ldots, b_n : 2^S \rightarrow \mathbb{R}$
	\item allocates disjoint sets $A_1, \ldots, A_n$ of items to
		bidders $1, \ldots, n$ (so $A_1, \ldots, A_n \subseteq S, A_i
		\cap A_j \neq \emptyset$ for $i \neq j$)
	\item charges bidder $i$ the price $p_i$
\end{itemize}

Note that the valuations $v_i : 2^S \rightarrow \mathbb{R}$ are objects of size
exponential in the size of $S$, or the number of items on offer.  We may
succinctly represent the bidders' valuations by a graph in which bidder $i$
wants to buy a path from vertex $s_i$ to vertex $t_i$. A path is a set of edges
$S'$. Consider the valuation:

\begin{equation*}
	v_i(S') = \begin{cases}
		\frac{1}{|S'|} & \text{$S'$ is a path from $s_i$ to $t_i$} \\
		0 & \text{otherwise}
	\end{cases}
\end{equation*}

\begin{definition}[Social welfare maximising auction mechanism]
	An auction mechanism is social welfare maximising if its allocation of
	items to bidders maximises (over all possible allocations) the social
	welfare $\sum_{i \in [n]} v_i(A_i)$
\end{definition}

\subsubsection{VCG mechanism}

The Vickrey-Clarke-Groves mechanism works as follows. It:

\begin{itemize}
	\item take bids $b_1, \ldots, b_n$
	\item compute an allocation $A_1, \ldots, A_n$ that maximises
		$\sum_i b_i(A_i)$ (``believe'' the bidders)
	\item allocate $A_1, \ldots, A_n$ to bidders $1, \ldots, n$
	\item for every bidder $k$, compute an allocation $A_1^{-k},
		\ldots, A_n^{-k}$ (an allocation excluding $k$) that
		maximises $\sum_{i: i \neq k} b_i(A_i^{-k})$
	\item charge bidder $k$ the amount $\sum_{i: i \neq k}
		b_i(A_i^{-k}) - \sum_{i: i \neq k} b_i(A_i)$ (each bidder
		is charged their \textsc{externality}
\end{itemize}

An externality is the decline in values of bidders other than $k$ resulting
from $k$ entering the auction. The amount each bidder is charged $\sum_{i: i
\neq k} b_i(A_i^{-k}) - \sum_{i: i \neq k} b_i(A_i)$ is a generalisation of
Vickrey's second price auction.

\subsubsection{Example}

Consider the following auction with 2 items and 4 bidders, with the following
valuations:

\begin{center}
	\begin{tabular}{|c|c|c|c|}
		\hline
		& $\{A\}$ & $\{B\}$ & $\{A,B\}$ \\ \hline
		$b_1$ & 0 & 0 & 5 \\ 
		$b_2$ & 2 & 0 & 0 \\ 
		$b_3$ & 0 & 1 & 0 \\ 
		$b_4$ & 0 & 1 & 1 \\ \hline
	\end{tabular}
\end{center}

It is easy to see that the allocation $(A_1, A_2, A_3, A_4) = (\{A,B\},
\emptyset, \emptyset, \emptyset)$ maximises $\sum_i b_i(A_i)$ (note that social
welfare and fairness are not necessarily aligned).

Now we calculate the amount that each bidder will be charged.  Since bidder 1
is the only player receiving any items, they will be the only one charged. Now,
$\sum_{i: i \neq 1} b_i(A_i) = b_2(A_2) + b_3(A_3) + b_4(A_4) = 0$. Next we
compute an allocation $(A_2, A_3, A_4)$ that maximises $\sum_{i: i \neq 1}
b_i(A_i^{-1})$. If we exclude player 1 from the auction, then $(A_2^{-1},
A_3^{-1}, A_4^{-1}) = (\{A\}, \emptyset, \{B\})$, and $b_2(\{A\}) +
b_3(\emptyset) + b_4(\{B\}) = 2+0+1 = 3$, so player 1 must pay $\sum_{i: i \neq
1} b_i(A_i^{-k}) - \sum_{i: i \neq 1} b_i(A_i) = 3 - 0$.

\begin{theorem}
	The VCG mechanism is incentive compatible and social welfare maximising --
	making bidders pay their externalities aligns their individual incentives
	with social welfare maximisation.
\end{theorem}

\begin{proof}
	Fix bidder $k$. The utility of bidder $k$ under allocation
	$A_k$ is:
	\begin{equation*}
		u_k(A_k) = v_k(A_k) - \left( \sum_{i: i \neq k}
		b_i(A_i^{-k}) - \sum_{i: i \neq k} b_i(A_i) \right)
	\end{equation*}

	Observation: the sum $\sum_{i: i \neq k} b_i(A_i^{-k})$ does not depend on
	bidder $k$. Hence bidder $k$ wants to maximise $v_k(A_k) + \sum_{i: i \neq
	k} b_i(A_i)$.

	\begin{claim}
		Bidding $b_k = v_k$ is a weakly dominant strategy for $k$.
	\end{claim}

	\begin{subproof}
		If $k$ bids $b_k = v_k$ hen the VCG mechanism produces the allocation
		$(A_1, \ldots, A_n)$ that maximises $\sum_i b_i(A_i) = b_k(A_k) +
		\sum_{i: i \neq k} b_i(A_i) = v_k(A_k) + \sum_{i: i \neq k} b_i(A_i)$.
		This is equal to the value that $k$ is trying to maximise, hence the
		VCG mechanism is working in $k$'s favour, so telling the truth will
		give $k$ the best outcome.
	\end{subproof}

	The VCG mechanism maximises social welfare by definition.
\end{proof}

\subsection{VCG mechanism's efficiency}

\subsubsection{Auctions with single-minded bidders}

In a combinatorial auction with single-minded bidders, each bidder is
interested only in one subset of item. Each bidder therefore needs only to
submit $(S_i, b_i)$, where $b_i$ is the bid for $S_i \subseteq S$, and $v_i(S')
= 0$ for all $S' \subseteq S$, $S' \neq S_i$. If bidder $i$ gets assigned its
desired set $S_i$, it is \emph{a} winner.

Consider the following example with 5 items and 7 single-minded bidders:

\begin{equation*}
	\begin{split}
		v_1(\{A,C,D\})   & = 7 \\
		v_2(\{B,E\})     & = 7 \\
		v_3(\{C\})       & = 3 \\
		v_4(\{A,B,C,D\}) & = 9 \\
		v_5(\{D\})       & = 4 \\
		v_6(\{A,B,C\})   & = 5 \\
		v_7(\{B,D\})     & = 5 \\
	\end{split}
\end{equation*}

We can allocate the items to players 1 and 2, which allocates all items and
achieves a social welfare of 14, or to players 2, 3, and 5, which allocates all
but item $A$ and achieves the same social welfare.

\begin{theorem}
	Finding an optimal allocation to bidders in the VCG mechanism (even in the
	case of single-minded bidders) is NP-hard.
\end{theorem}

First, recall the problems \textsc{MaxIndependentSet}: given a graph $G =
(V,E)$, we wish to compute a subset of maximum cardinality $S \subseteq V$ such
that for all $u,v \in V$, $(u,v) \not \in E$.

\begin{proof}
	We will prove it by constructing a polynomial-time reduction from
	\textsc{MaxIndependentSet} in graphs. To prove that computing an allocation
	(\textsc{CAwSMB}) is NP-hard, we have to show that, given an instance of
	\textsc{MaxIndependentSet}, we can transform it into an instance of our
	combinatorial auction (i.e., we are showing that \textsc{MaxIndependentSet}
	$\le^P$ \textsc{CAwSMB}).

	Let $G=(V,E)$ be an undirected graph. Construct a combinatorial auction by
	setting the bidders as the vertices, edges as the items, and bid for player
	$i$ as $(\{e \in E \, : \, e$ is incident to $v \}, 1)$.

	A set $W \subseteq V$ is a set of winners iff $W$ is an independent set in
	$G$. Similarly, a set $W \subseteq V$ is social welfare maximising iff $W$
	is a maximum independent set. This reduction can obviously be done in
	polynomial time.
\end{proof}

The VCG mechanism has good theoretical properties such as dominant strategy
incentive compatibility, however (as far as we know so far) it is
computationally intractable. The death blow for the VCG mechanism comes in the
following theorem.

\begin{theorem}
	\label{thm:maxIndSetInapproximability}
	For every small constant $\varepsilon >0$, computing an
	$|E|^{\frac{1}{2} - \varepsilon}$-approximate (i.e.
	anything better than $\sqrt{|E|}$) solution of
	\textsc{MaxIndependentSet} cannot be done in polynomial
	time, under standard complexity-theoretic assumptions
	(e.g., that NP $\not \subseteq$ BPP).
\end{theorem}

The best hope we can have is to design an efficient auction mechanism that
produces $\sqrt{|S|}$-approximate allocations of items to bidders. If the sizes
of sets that bidders are interested in is at most 2, then the problem is easy:
it is simply a matching problem. If they are at least 3 however, hardness kicks
in and the problem is intractable.

\begin{fact}
	If we replace optimal allocations in the VCG mechanism by approximate ones
	then the resulting mechanism is not incentive compatible.
\end{fact}

\subsection{A greedy mechanism for auctions with single-minded bidders}

In this section we design a greedy mechanism for computing optimal allocations
for auctions with single-minded bidders. The advantage of such a mechanism is
that it is computationally efficient, incentive compatible. A possible drawback
is that it ``only'' computes a $\sqrt{|S|}$-approximate social welfare
maximising allocation.

An auction mechanism is \emph{monotone} if any bidder who bids $(S_i, b_i)$ and
wins (i.e. is allocated $S_i$) also wins if they bid $(A, w)$, with $A
\subseteq S_i$ and $w \ge b_i$ (they win if they bid more for less).

\begin{definition}[Critical value pricing]
	In a mechanism with critical value pricing, a winner $i$ who bids $(S_i,
	b_i)$ pays the minimum price needed to win (the infimum of all $w$ such
	that the bid $(S_i, w)$ would also win).
\end{definition}

\begin{lemma}
	\label{lem:singleMindedDSIC}
	A mechanism for auctions with single-minded bidders in which losers pay 0
	is incentive compatible iff it is monotone and it uses critical value
	pricing.
\end{lemma}

The greedy mechanism:
\begin{itemize}
	\item Winner determination:
		\begin{itemize}
			\item Sort bids and rename such that
				$\frac{b_1}{\sqrt{|S_1|}} \ge \ldots \ge
				\frac{b_n}{\sqrt{|S_n|}}$
			\item initialise $W \leftarrow \emptyset$
			\item for $i = 1, \ldots, n$, if $S_i \cap (\bigcup_{j \in W} S_j)
				= \emptyset$, then $W = W \cup \{i\}$ (i.e. if all
				of $i$'s desired items are available)
		\end{itemize}
	\item Critical value pricing
		\begin{itemize}
			\item winner $i$ pays $\frac{v_j}{\sqrt{|S_j|/|S_i|}}$,
				where $j$ is the smallest index greater than $i$ such
				that for all $k < j$, either $k=i$ or $S_k \cap S_j =
				\emptyset$
			\item winner $i$ pays 0 if no such $j$ exists
		\end{itemize}
\end{itemize}

The above winner-determination rule is monotone as increasing bid $b_i$ or
decreasing $S_i$ will make the ratio $\frac{b_i}{\sqrt{|S_i|}}$ larger.
Similarly, the payment rule indeed uses critical-value pricing as $i$ wins iff
$i$ appears before $j$.

\begin{theorem}
	The greedy mechanism is incentive compatible and produces a
	$\sqrt{|S|}$-approximation allocation w.r.t. social welfare.
\end{theorem}

\begin{proof}
	The mechanism's incentive compatibility follows from
	Lemma~\ref{lem:singleMindedDSIC}, which states that a mechanism for
	auctions with single-minded bidders in which losers pay 0 is DSIC iff it is
	monotone and it uses critical-value pricing.  To prove the approximation
	bound, we want to show:

	\begin{equation*}
		\sum_{i \in W^*} v_i \le \sum_{i \in W} v_i
	\end{equation*}

	where $W^*$ is the set of winners in the welfare-maximising solution, and
	$W$ is the set of winners produced by the greedy solution.

	Let $W^*_i = \{ j \in W^*, j \ge i \, : \, S_i \cap S_j \neq \emptyset \}$,
	i.e. bidders that win in an optimal allocation but cannot in the greedy
	auction as their desired set conflicts with that of bidder $i$, who appears
	before them. For $j \in W^*_i$, we have $b_j \le \sqrt{|S_j|} \cdot
	\frac{b_i}{\sqrt{|S_i|}}$. Hence,

	\begin{equation*}
		\sum_{j \in W^*_i} b_j \le \frac{b_i}{\sqrt{|S_i|}} \cdot
		\sum_{j \in W^*_i} \sqrt{|S_j|}
	\end{equation*}

	\begin{claim}
		$\sum_{j \in W^*_i} \sqrt{|S_j|} \le \sqrt{|S_i|} \cdot
		\sqrt{|S|}$
	\end{claim}

	\begin{subproof}
		The Cauchy-Schwarz inequality states that $(\sum_{k=1} x_k y_k)^2 \le
		\sum_k x_k^2 \cdot \sum_k y_k^2$. Choosing $x_k = 1$ and $y_k =
		\sqrt{|S_k|}$, we get

		\begin{equation*}
			(\sum_{k=1} \sqrt{|S_k|})^2 \le |W^*_i| \cdot \sum_{j
			\in W^*_i} |S_j|
		\end{equation*}

		Taking the square root of both sides gives us
		\begin{equation*}
			\sum_{k=1} \sqrt{|S_k|} \le \sqrt{|W^*_i|} \cdot
			\sqrt{\sum_{j \in W^*_i} |S_j|}
		\end{equation*}

		Observe that $|W^*_i| \le |S_i|$, as for every $j \in W^*_i$, $S_j$
		intersects $S_i$ and all of these intersections are disjoint. We also
		have $\sum_{j \in W^*_i} |S_j| \le |S|$.
	\end{subproof}

	Using this claim and summing over all $i \in W$, we get
	\begin{equation*}
		\begin{split}
			\sum_{i \in W} \sum_{j \in W^*_i} b_j & \le \sum_{i \in W}
			\frac{b_i}{\sqrt{|S_i|}} \cdot \sum_{j \in W^*_i}
			\sqrt{|S_j|} \\
			& \le \sum_{i \in W} \frac{b_i}{\sqrt{|S_i|}}
			\cdot \sqrt{|S_i|} \cdot \sqrt{|S|} \\
			& \le \sqrt{|S|} \cdot \sum_{i \in W} b_i
		\end{split}
	\end{equation*}

	As $W^* \subseteq \bigcup_{i \in W} W^*_i$, we have that $\sum_{i \in W}
	\sum_{j \in W^*_i} b_j$ is an upper bound of the optimal solution welfare.
\end{proof}

We conclude that the greedy auction mechanism is incentive compatible and
approximates the optimal social welfare within a factor of $\sqrt{|S|}$. Since
computing an allocation of items to bidders that gives a $|S|^{\frac{1}{2} -
\varepsilon}$-approximation in polynomial time under reasonable complexity
theoretic assumptions is not possible by
Theorem~\ref{thm:maxIndSetInapproximability}, this is essentially the best
bound we can hope for.
