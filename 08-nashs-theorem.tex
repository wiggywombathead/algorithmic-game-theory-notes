\section{Nash's Theorem}

We prove that every finite game has a Nash Equilibrium by drawing together two
seemingly unrelated things.

\subsection{Sperner's Lemma}

Consider the graph formed from the triangulation of a triangle -- that is, a
graph that is made up only of smaller triangles.

\begin{definition}[Valid colouring]
	A valid colouring is a colouring of the vertices, each with a colour in
	the set $\{R,G,B\}$, such that:
	\begin{itemize}
		\item each of the three corners is a distinct colour
		\item no vertex on the side of the big triangle has the same colour
			as the opposite corner
	\end{itemize}
\end{definition}

We call a triangle \textit{panchromatic} if each of its vertices have a unique
colour. We will now state Sperner's Lemma and prove it in two different ways.

\begin{theorem}[Sperner's Lemma]
	Every valid colouring has an odd number of panchromatic triangles.
\end{theorem}

\begin{proof}[Proof 1 \emph{(by counting RB triangles)}]
	Let:
	\begin{itemize}
		\item $\bigtriangleup^{RGB} :=$ number of RGB triangles
		\item $\bigtriangleup^{RB} :=$ number of RB triangles
		\item $|^{RB}_X :=$ number of external RB edges, $|^{RB}_I :=$
			number of internal RB edges
	\end{itemize}

	By an external edge, we mean an edge that was formed directly from the
	original triangle, and an internal edge is every edge contained inside the
	triangulation.

	\begin{claim*}
		$|^{RB}_X$ is odd.
	\end{claim*}
	\begin{subproof}
		Consider two corners of the large triangle, where the first is coloured
		red, and the other blue. Since we have a valid colouring, no vertex
		along the edges joining these two corners is coloured green, so we only
		have RB, RR, or BB edges. We get a new RB edge whenever we change
		colour. Since the colours of the corners are different, we must change
		colour an odd number of times, so the number of RB edges on the
		exterior is odd.
	\end{subproof}

	\begin{claim*}
		Each RGB triangle has one RB edge. Each RB triangle has two RB edges.
		All other triangles have 0 RB edges.
	\end{claim*}

	\begin{claim*}
		$\bigtriangleup^{RGB} + 2\bigtriangleup^{RB} = |^{RB}_X +
		2|^{RB}_I$
	\end{claim*}
	\begin{subproof}
		The number of RB edges is equal to
		$\bigtriangleup^{RGB}+2\bigtriangleup^{RB}$. This counts all the
		interior RB edges twice, as each internal edge separates two triangles,
		and each external RB edge once, as exactly one triangle touches it.
	\end{subproof}

	Hence $\bigtriangleup^{RGB} = |^{RB}_X + 2|^{RB}_I - 2\bigtriangleup^{RB} =
	2(|^{RB}_I - \bigtriangleup^{RB}) + |^{RB}_X$, which is odd.

\end{proof}

\begin{proof}[Proof 2 \emph{(by following the End-of-a-Line)}]
	Construct a directed graph based on the triangulation as follows:
	\begin{itemize}
		\item add edges from the B corner to all vertices on the RB side
		\item create a node for each cell (internal triangle), and one for
			the ``outside'' edge
		\item create a directed arc across every RB edge, where R is ``to
			the left'' as we traverse the arc)
	\end{itemize}

	Every node has in-degree at most one and out-degree at most one. Thus the
	graph consists only of isolated vertices, edge-disjoint paths, and
	edge-disjoint cycles. By the Handshaking Lemma, the number of nodes of
	degree exactly one is even. Nodes with degree exactly one appear in
	panchromatic triangles (one way in, no way out), and only RGB triangles
	contain these nodes of degree exactly one. We began by creating an
	artificial node outside of the large triangle -- every other node of degree
	one corresponds to a panchromatic triangle. Hence the number of nodes of
	degree exactly one in the original graph is odd, so the number of RGB
	triangles in the original graph is odd.
\end{proof}

\subsection{Brouwer's Fixed Point Theorem}

A $k$-simplex is a $k$-dimensional polytope that is the convex hull of its
$k+1$ vertices. The standard simplex that we will make use of is a simplex
formed from $k+1$ standard unit vectors:
\begin{equation}
	S^d = \{ (x_1, \ldots, x_{d+1}) \, | \, x_1 + \ldots + x_{d+1} = 1, x_i
	\ge 0 \text{ for } i = 1, \ldots, d+1 \}
\end{equation}

A set is compact if it is \textit{closed} -- it contains all its limit points
-- and \textit{bounded} -- all its points lie within some fixed distance from
one another.

A convex set is a set of points such that, given any two points in the set, the
line joining them lies entirely within that set.

A fixed point of a function $f: A \rightarrow B$ is an element of the
function's domain that maps to itself, that is, $c \in A$ is a fixed point of
$f$ if $f(c) = c$.

\begin{theorem}[Brouwer's Fixed Point Theorem]
	If $f: S \rightarrow S$ is a continuous function from a compact convex set
	to itself, then $f$ has a fixed point.
\end{theorem}

\begin{proof}[Proof \emph{(for d=1)}]
	$S^1$ is the unit interval. Let $f:[0,1] \rightarrow [0,1]$ be a continuous
	function. If $f(0)=0$ or $f(1)=1$ then we have found our fixed point.
	Otherwise, we have $f(0)>0$ and $f(1)<1$. The function $g(x) := x - f(x)$
	is continuous, and $g(0)<0$ and $g(1)>0$. By the Intermediate Value
	Theorem, there must be some $x \in [0,1]$ for which $g(x) = 0$. Hence $x =
	f(x)$.
\end{proof}

\begin{proof}[Proof \emph{(for d=2)}]
	The function $f$ sends point $(\alpha, \beta, \gamma)$ to another point
	in the triangle: $f(\alpha, \beta, \gamma) = (\conj{\alpha},
	\conj{\beta}, \conj{\gamma})$. We begin the proof by defining a valid
	colouring of $S^2$ based on $f$:
	\begin{itemize}
		\item R if $\alpha > \conj{\alpha}$
		\item B if $\alpha \le \conj{\alpha}$ and $\beta > \conj{\beta}$
		\item G if $\alpha \le \conj{\alpha}$ and $\beta \le \conj{\beta}$
			and $\gamma < \conj{\gamma}$
	\end{itemize}

	\begin{fact}
		If a point has no colour, then it is a fixed point.
	\end{fact}

	\begin{claim*}
		The colouring is valid.
	\end{claim*}
	\begin{subproof}
		Recall that a colouring is valid if each of the three vertices of a
		triangle receives a different colour and no point on the side is
		assigned the colour equal to the colour on the opposite corner. The
		colours of the points $(1,0,0), (0,1,0),$ and $(0,0,1)$ are R, B,
		and G respectively. Furthermore,
		\begin{itemize}
			\item no point on the $(1,0,0)-(0,1,0)$ edge has colour G
			\item no point on the $(0,0,1)-(1,0,0)$ edge has colour B
			\item no point on the $(0,1,0)-(0,0,1)$ edge has colour R
		\end{itemize}
	\end{subproof}

	We then use Sperner's Lemma and the compactness of $S^2$ to find a fixed
	point. Let $T_1, T_2, \ldots$ be triangulations of $S^2$ such that the
	largest diameter of the cells converges to 0. By Sperner's Lemma, every
	$T_i$ has a panchromatic cell with corners:
	\begin{itemize}
		\item $x_i = (\alpha_i, \beta_i, \gamma_i)$ with colour R
		\item $x_i' = (\alpha_i', \beta_i', \gamma_i')$ with colour B
		\item $x_i'' = (\alpha_i'', \beta_i'', \gamma_i'')$ with colour G
	\end{itemize}

	In other words, assuming that we have:
	\begin{equation}
		\begin{split}
			f(x_i) & = (\conj{\alpha}_i, \conj{\beta}_i, \conj{\gamma}_i) \\
			f(x_i') & = (\conj{\alpha}_i', \conj{\beta}_i', \conj{\gamma}_i') \\
			f(x_i'') & = (\conj{\alpha}_i'', \conj{\beta}_i'', \conj{\gamma}_i'')
		\end{split}
	\end{equation}

	Then:
	\begin{equation}
		\label{eq:fixedPoints}
		\begin{split}
			\alpha_i > \conj{\alpha}_i & \text{ (as $x_i$ is R)} \\
			\beta_i' > \conj{\beta}_i' & \text{ (as $x_i'$ is B)} \\
			\gamma_i'' > \conj{\gamma}_i'' & \text{ (as $x_i''$ is G)}
		\end{split}
	\end{equation}

	Since $S^2$ is a compact (closed and bounded) subset of $\mathbb{R}^3$, by
	the Bolzano-Weierstrass Theorem\footnote{Each bounded sequence in
	$\mathbb{R}^n$ has a convergent subsequence.}, there is a sequence $i_1 <
	i_2 < \ldots$ such that $\lim_{k \rightarrow \infty} x_{i_k}$ exists (the
	sequence $\langle x_{i_k} \rangle$ converges).

	Since the diameters of the cells in triangulations $T_{i_1}, T_{i_2},
	\ldots$ converges to 0, we have $\lim_{k \rightarrow \infty} x_{i_k} =
	\lim_{k \rightarrow \infty} x_{i_k}' = \lim_{k \rightarrow \infty}
	x_{i_k}''$. Let $z = (\alpha, \beta, \gamma)$ and $f(z) =
	(\conj{\alpha}, \conj{\beta}, \conj{\gamma})$.
	Using~\eqref{eq:fixedPoints}, we have $\alpha \ge \conj{\alpha}, \beta
	\ge \conj{\beta}, \gamma \ge \conj{\gamma}$. Thus it follows that
	$\alpha = \conj{\alpha}, \beta = \conj{\beta}, \gamma = \conj{\gamma}$
	since $\alpha + \beta + \gamma = 1$ and $\conj{\alpha} + \conj{\beta} +
	\conj{\gamma}$, so we have $f(z) = z$, a fixed point.
\end{proof}

\begin{claim}
	Brouwer's Theorem can be proved for $d \ge 3$ by induction.
\end{claim}

\begin{remark}
	Brouwer's Fixed Point Theorem holds for continuous functions $f:C
	\rightarrow C$ where $C$ is convex and compact. Any convex compact set can
	be continuously deformed into $S^d$ for some $d$.
\end{remark}

\subsection{Nash's Theorem}

We now prove the big Nash's Theorem for two players by bringing together the
previous two results. The proof for $n$ players is a straightforward
generalisation. Note that by a finite game, we mean a game in which there is a
finite number of players each with a finite number of pure strategies.

\begin{theorem}[Nash, 1951]
	Every finite game has a Nash Equilibrium.
\end{theorem}

\begin{proof}[Proof of Nash's Theorem]
	Player I has payoff matrix $\vect{A}$ and plays mixed strategy
	$\vect{x} \in \Delta^{m_1}$, while player II has payoff matrix
	$\vect{B}$ and plays mixed strategy $\vect{y} \in \Delta^{m_2}$.

	Let $k_i(x,y)$ denote player I's gain in utility from switching to
	strategy $i \in S_\text{I}$, and $k_j'(x,y)$ denote player II's gain in
	utility from switching to strategy $j \in S_\text{II}$:
	\begin{equation}
		\begin{split}
			k_i(x,y) & := \max \{ 0, (\vect{Ay})_i - \vect{x^\top A y} \} \\
			k_j'(x,y) & := \max \{ 0, (\vect{x^\top B})_j - \vect{x^\top B y} \} \\
		\end{split}
	\end{equation}

	We are at a Nash Equilibrium when $k_i(x,y) = k_j'(x,y)$ for some
	strategies $i \in S_\text{I}, j \in S_\text{II}$. Define the following
	mixed strategy for player I:
	\begin{equation}
		g(x,y) := \frac{1}{1 + \sum_{i \in [m_1]} k_i(x,y)} \begin{pmatrix}
			x_1 + k_1(x,y) \\
			\vdots \\
			x_{m_1} + k_{m_1}(x,y)
		\end{pmatrix}
	\end{equation}

	And the same for player II:
	\begin{equation}
		h(x,y) := \frac{1}{1 + \sum_{j \in [m_2]} k_j'(x,y)} \begin{pmatrix}
			y_1 + k_1'(x,y) \\
			\vdots \\
			y_{m_2} + k_{m_2}'(x,y)
		\end{pmatrix}
	\end{equation}

	\begin{claim*}
		$g(x,y)$ and $h(x,y)$ are valid mixed strategies.
	\end{claim*}
	\begin{subproof}
		We will prove for $g(x,y)$, and the same reasoning applies for
		$h(x,y)$.  Let $\conj{x} := \begin{pmatrix}
			x_1 + k_1(x,y) \\
			\vdots \\
			x_{m_1} + k_{m_1}(x,y) \\
		\end{pmatrix}$. In a valid mixed strategy the sum of entries must
		be equal to 1. We have:
		\begin{equation*}
			\begin{split}
				\sum_i \conj{x}_i & = \sum_i (x_i + k_i(x,y)) = \sum_i x_i +
				\sum_i k_i(x,y) \\
				& = 1 + \sum_i k_i(x,y)
			\end{split}
		\end{equation*}

		Hence, $\frac{1}{1 + \sum_i k_i(x,y)} \sum_i \conj{x}_i = 1$ and
		each $\conj{x}_i \ge 0$, since $k_i(x,y) \ge 0$ by definition.
	\end{subproof}

	Now define the continuous function $f:\Delta^{m_1} \times
	\Delta^{m_2} \rightarrow \Delta^{m_1} \times \Delta^{m_2}$ as
	follows:
	\begin{equation}
		f(x,y) = (g(x,y), h(x,y))
	\end{equation}

	Since $\Delta^{m_1} \times \Delta^{m_2}$ is a convex compact set,
	Brouwer's Theorem implies that $f$ has a fixed point. That is, there is
	an $x^* \in \Delta^{m_1}$ and a $y^* \in \Delta^{m_2}$ such that
	$f(x^*,y^*) = (x^*,y^*)$. Now we just need to show that $(x^*,y^*)$ is
	a Nash Equilibrium.

	\begin{claim*}
		$(x^*, y^*)$ is a Nash Equilibrium.
	\end{claim*}
	\begin{subproof}
		We must show that for all $i \in [m_1], j \in [m_2], k_i(x^*,y^*) =
		k_j'(x^*,y^*) = 0$. Assume there is an $i \in [m_1]$ such that
		$k_i(x^*,y^*) > 0$. Then there must be a $j \in [m_1]$ such that
		$x_j \neq 0$ and $k_j(x^*, y^*) = 0$. If we suppose that this is
		not true (i.e. that every $k_j(x^*,y^*) > 0$), then we have:
		\begin{equation*}
			\vect{x^\top A y} = \sum_{\ell \in [m_1]} x_\ell
			(\vect{Ay})_\ell > \sum_{\ell \in [m_1]} x_\ell (\vect{x^\top A
			y}) = \vect{x^\top A y} \sum_{\ell \in [m_1]} x_\ell =
			\vect{x^\top A y}
		\end{equation*}

		A contradiction.
	\end{subproof}

	Since $(x^*,y^*)$ is a fixed point of $f$, $g(x^*,y^*) = x^*$.
	Therefore,
	\begin{equation*}
		\frac{x^*_i + k_i(x^*, y^*)}{1 + \sum_{\ell \in [m_1]} k_\ell (x^*,
		y^*)} = x^* \qquad (\neq 0 \text{ as $x^*_i + k_i(x^*, y^*) \neq 0$})
	\end{equation*}

	Rearranging, we have $\sum_{\ell \in [m_1]} k_l(x^*, y^*) =
	\dfrac{k_i(x^*, y^*)}{x^*_i} > 0$, but then:
	\begin{equation*}
		\begin{split}
			\frac{x^*_j + k_j(x^*,y^*)}{1 + \sum_{\ell \in [m_1]} k_\ell
			(x^*, y^*)} & = \frac{x^*_j}{1 + \sum_{\ell \in [m_1]} k_\ell
			(x^*, y^*)} \\
			& = \frac{x^*_j}{1 + k_i(x^*,y^*) / x^*_i} \neq x^*_j
		\end{split}
	\end{equation*}

	This contradicts the fact that $(x^*,y^*)$ is a fixed point, and hence
	there is no $i$ such that $k_i(x^*, y^*) = 0$. Similar reasoning shows
	that $k_j'(x^*,y^*) = 0$ for all $j \in [m_2]$. Therefore no player can
	improve their expected payoff by switching strategy, so no player the
	incentive to unilaterally deviate. Thus $(x^*,y^*)$ is a Nash
	Equilibrium.
\end{proof}
