\section{Vector-matrix notation for 2-player games}
	Let the payoffs to player I be denoted by the matrix $\vect{A}$, where
	entry $\vect{A}_{ij}$ gives the payoff to player I when playing strategy $i
	\in S_\text{I}$ given that player II is playing strategy $j \in
	S_\text{II}$. Define player II's payoff matrix $\vect{B}$ in a similar
	fashion, where the rows still represent player I's strategies and the
	columns player II's.

	Let $\vect{x} \in \Delta^{m_1}$ denote player I's mixed strategy, and
	$\vect{y} \in \Delta^{m_2}$ denote player II's. Then player I's expected
	payoff from playing pure strategy $i \in S_\text{I}$ is given by the entry
	$(\vect{Ay})_i$, and player II's expected payoff from playing pure strategy
	$j \in S_\text{II}$ is given by $(\vect{x^\top B})_j$.

	Note when computing player II's expected utility each entry in the
	resulting vector is for some fixed column, while when computing player I's
	then each entry is for some fixed row.

	\subsection{Example}
		Suppose we have the following two-player zero-sum (i.e. $\vect{B} =
		-\vect{A}$) game:
		\begin{equation*}
			\vect{A} = 
			\begin{pmatrix}
				28 & 1 & -38 & -11 \\
				4 & 3 & 2 & -3 \\
				5 & -3 & 4 & 3 \\
				-19 & -9 & 29 & 1
			\end{pmatrix},
			\vect{x} =
			\begin{pmatrix}
				0 \\
				\frac{1}{2} \\
				\frac{1}{2} \\
				0
			\end{pmatrix},
			\vect{y} = 
			\begin{pmatrix}
				0 \\
				\frac{1}{2} \\
				0 \\
				\frac{1}{2}
			\end{pmatrix}
		\end{equation*}

		In this example, player I's expected utilities $\vect{A} \vect{y}
		= \begin{pmatrix} -5 & 0 & 0 & 4 \end{pmatrix}$ and player II's expected
		utilities $\vect{x}^\top \vect{B} = \begin{pmatrix} -\frac{9}{2} & 0
		& -3 & 0 \end{pmatrix}$. Hence pure strategies 2 and 3
		are best responses for player I to $\vect{y}$ (as they yield the
		highest expected utility, 0), and furthermore player I is indifferent
		to them. Similarly pure strategies 2 and 4 are best responses for
		player II to $\vect{x}$, and is also indifferent to them.

		Note that:
		\begin{equation*}
			\vect{x} =
			\begin{pmatrix}
				0 \\
				\frac{1}{2} \\
				\frac{1}{2} \\
				0
			\end{pmatrix},
			\vect{A} \vect{y} =
			\begin{pmatrix}
				-5 \\
				0 \\
				0 \\
				-4
			\end{pmatrix}
		\end{equation*}

		Thus $\vect{x}$ is a best response to $\vect{y}$ (strategies 2 and 3
		for I are best responses to $\vect{y}$, and player I mixes between
		strategies 2 and 3). Similarly:
		\begin{equation*}
			\vect{y} =
			\begin{pmatrix}
				0 \\
				\frac{1}{2} \\
				0 \\
				\frac{1}{2} \\
			\end{pmatrix},
			\vect{x}^\top \vect{B} = 
			\begin{pmatrix}
				-\frac{9}{2} \\
				0 \\
				-3 \\
				0 \\
			\end{pmatrix}
		\end{equation*}

		Hence $\vect{y}$ is a best response to $\vect{x}$ (strategies 1 and
		3 are best responses, and II mixes between strategies 1 and 3).
