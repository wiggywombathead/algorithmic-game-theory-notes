\section{Linear Programming and LP Duality}

A linear program is a linear optimisation problem -- we are maximising
(minimising) a linear objective function, subject to a set of linear
constraints (inequalities).

\begin{definition}[Linear Program]
	Given a matrix $\vect{A} \in \mathbb{R}^{m \times n}$ and vectors $\vect{b}
	\in \mathbb{R}^m, \vect{c} \in \mathbb{R}^n$, we want to find a vector
	$\vect{x} \in \mathbb{R}^n$ such that:

	\begin{equation}
		\begin{split}
			\vect{x} \in \argmax \vect{c}^\top \vect{x} \\
			\vect{A}\vect{x} \le \vect{b} \\
			\vect{x} \ge \vect{0}
		\end{split}
	\end{equation}
\end{definition}

\subsection{Taking the Dual}

There are four steps to taking the dual: swap maximise with minimise, swap
vectors $\vect{b}$ and $\vect{c}$, transpose $\vect{A}$, and change the
direction of the (non-trivial) inequalities. In the dual there is one variable
for every constraint, and one constraint for every variable. Hence for a primal
LP with $n$ variables and $m$ constraints, the dual LP will have $m$ variables
and $n$ constraints.

Consider the primal linear program, P:

\begin{equation}
	\begin{split}
		\text{maximise } \vect{c}^\top \vect{x} \text{ subject to }
		\vect{A}\vect{x} & \le \vect{b} \\
		\vect{x} & \ge \vect{0}
	\end{split}
\end{equation}

The dual of the linear program P, D, is as follows:

\begin{equation}
	\begin{split}
		\text{minimise } \vect{b}^\top \vect{y} \text{ subject to }
		\vect{A}^\top \vect{y} & \ge \vect{c} \\
		\vect{y} & \ge \vect{0}
	\end{split}
\end{equation}

\begin{theorem}[Strong Duality]
	If P or D has an optimal solution of finite value, then so too does the
	other, and the value of the objective function of the optimal solutions are
	equal.
\end{theorem}
